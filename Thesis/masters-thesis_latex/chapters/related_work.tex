\chapter{Related Work}
\label{chap:related_work}
This Chapter explores the related work on encrypted traffic-analysis attacks in various contexts. We separate traffic-analysis attacks into 3 categories \textit{Website Fingerprinting}, \textit{Mobile Application Fingerprinting} and \textit{Wearable Devices Fingerprinting}.

\paragraph{Website Fingerprinting (WF).} WF consists of identifying web pages from their encrypted traffic traces. WF has been broadly studied under diverse assumptions. Early work on WF started in 1998 on HTTP 1.0 over SSL by leveraging HTTP objects count and sizes~\cite{cheng1998traffic, 10.5555/829514.830535}.  Later on, with HTTP 1.1, persistent connection was used making resources isolation not possible. In 2006, Levine et al. proposed an attack over 2,000 web pages only based on packet sizes discarding the assumption of isolating resources~\cite{liberatore2006inferring}. In 2011, Panchenko et al. constrained the assumption of observable packet length by constructing an attack on the Tor web browser~\cite{10.1145/2046556.2046570}. After that, many researchers further improved the results from Panchenko et al. on Tor by considering different models \cite{cai2012touching, 10.1145/2517840.2517851, 184463}. In 2016, Wang et al. investigated an attack under various more realistic conditions such as an attacker who does not know when traffic of web pages starts and ends \cite{OnRealisticallyAttackingTorwithWebsiteFingerprinting}. In a concurrent work, Hayes et al. provided a new WF attack by using an adapted version of RF to generate fingerprint. In their work, they identified 30 monitored web pages out of 100,000 unmonitored web pages which pushed away the closed-world assumption where the victim is only visiting a constrained number of web pages~\cite{k-fingerprinting}. Nowadays, the state-of-the-art WF on Tor are based on Convolutionnal and Deep Neural Network model, archiving around 98\% to 99\% accuracy~\cite{he2018deep, bhat2019var, rahman2019tik}. 



\paragraph{Mobile Application Fingerprinting (MAF).} With the increase in popularity of smartphones, researchers also studied MAF.  Which consists of fingerprinting mobile applications and actions from the traffic they generate. Perhaps the first work to target multiple mobile applications was due to Conti et al. in 2015~\cite{conti2015can}. In their work, they fingerprinted 17 actions from across different applications. Later on, they extended their first paper to provide an attack that works on seven different applications and multiple actions per application~\cite{conti2015analyzing}. Then, researchers improved their results in terms of number of application fingerprinted. In this way, a system called Appscanner, with automatic dataset generation capabilities using \texttt{adb} and \texttt{monkeyrunner}\footnote{We also used adb and monkeyrunner for our automation system. However, our system is significantly different, as it works on smartwatches and we collect traces over Bluetooth.} could successfully fingerprint up to 110 applications using Random Forest~\cite{taylor2016appscanner}. However, criticizes emerged as their system was not robust over time and across different OS versions~\cite{taylor2017robust}. More recently, Wang et al. proposed an automation tool to generate the dataset from real 28 users over two periods of 4 and 10 weeks. They achieved 91.8\% accuracy on a 1D CNN classifier over 142 application~\cite{wang2020real}.

\newpage


\paragraph{IoT Devices Fingerprinting (IDF).} Das et al. inferred fitness tracker devices over Bluetooth LE, by listening on advertising channels~\cite{das2016uncovering}. In their paper, they also classified four types of physical activities amongst 10 persons from encrypted BLE traffic. Acar et al. achieved 90\% accuracy on fingerprinting diverse IoT devices and their respective state/activity on BLE and two other protocol~\cite{acar2018peek}. Aksu et al. identifies models of smartwatches devices using Inter-Arrival-Time (IAT) of Bluetooth packet~\cite{fingerprintWearableDevices}. However, their approach implies a Bluetooth sniffer directly inside the smartphone. Our attack uses a Bluetooth Recorder external to the smartphone.  





\paragraph{Differentiation from previous works.} Our work has several important differences compared to the aforementioned works. First and foremost, WF and MAF both performs traffic-analysis on top of TCP/IP packets, hence the network stack is fairly different. WF targets web pages, for which it is much easier to build large dataset. Moreover, compared to WF, there is no mechanism such as Tor that hides packet length, thus our attack can fully exploit features based on packet length. While MAF also targets applications, the work presented take advantages of particularities in TCP/IP that we cannot exploit with Bluetooth (such as TCP re-transmission flags or IP destination address). Fingerprinting wearables is a relatively new field, and while BLE showed obvious privacy leakage on advertising channels, our work focus on Bluetooth Classic. Finally, none of the work we presented was intended to target smartwatch applications over Bluetooth. 



