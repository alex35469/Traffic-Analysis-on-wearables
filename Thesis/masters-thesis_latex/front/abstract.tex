\chapter{Abstract}
Smartwatches store, process, and communicate a lot of personal and sensitive information (notably related to the owner's lifestyle and health). In this thesis, we explore the feasibility of inferring sensitive information from encrypted Bluetooth traffic between a smartwatch and its paired smartphone. We consider a passive eavesdropper monitoring the encrypted Bluetooth transmissions, without breaking the Bluetooth encryption scheme nor interacting with the devices, and we show that such attacker can infer sensitive information such as the application currently running and actions performed by the user. To carry out the attack, we first build a system that generates data at large scale by automating actions on the smartwatches. Then, we design a Machine-Learning based model that infers application usage on smartwatches. We analyze relevant features for application and action identification, and find out that packet inter-arrival-time, sizes and directions reveal enough information for the attack to be successful. We show how our model generalizes between smartwatches of different vendors, and how it can be used to perform long-term activity tracking (over the course of a day).
